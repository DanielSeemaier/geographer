\documentclass[a4paper,10pt]{article}

\usepackage[a4paper, total={6in, 9in}]{geometry}
\usepackage[utf8]{inputenc}
\usepackage{amsmath}
\usepackage{amsfonts}
\usepackage{graphicx}
\usepackage[ruled,vlined]{algorithm2e}
\usepackage{hyperref}
\usepackage{cleveref}
\usepackage{color}
\usepackage{subcaption}

\author{Harry Tzovas}

\definecolor{blue1}{RGB}{44,127,184}

\newcommand{\wave}{$\mathcal{WAVE}$ }

\newcommand{\bull}{$\bullet$}
\newcommand{\red}[1]{{\color{red}{#1}}}
\newcommand{\blue}[1]{{\color{blue1}{#1}}}
\newcommand{\mz}{\mathbb{Z}}
\newcommand{\quot}[1]{``#1''}
\newcommand{\km}{$k$-means}
\newcommand{\todo}[1]{{\red{TODO}}: #1}
\newcommand\noIndent[1]{
  \par\vbox{\parbox[t]{\linewidth}{#1}}
}

\graphicspath{{/home/harry/wave/figures/}}



\begin{document}

This is meant as a place to document problems and TODOs, possible solutions and ideas and, when some
idea is implemented, what are the results.

\section*{PE renumbering}
Given a partition, we want to redistribute data based on that. A function for this  exists in
aux::redistributeFromPartition. Theoretically, it can happen that PE $P_j$ have local points that belong
to other PEs, that is, $\text{localPart}(i)\neq j,\; \forall i\in P_j$. So it will send local data and receive new.

\paragraph*{Idea:}  Before redistribution, every PE goes over its local points and counts how many points
are belong to each block. Then, if for PE $P_j$, the majority of its local points belong to block $k$,
we will renumber block $k$ to $j$. This way, $P_j$ will keep as many points as possible. 

The problem is that many PEs can request for different renumbering of the same block. 
For example, in PE $P_0$ the majority of its local points, say $30\%$ belong to block $3$ 
but also, the majority of points in $P_1$ belongs also block $3$ with $40\%$ of the points in $P_1$. 
So $P_0$ want the renumbering 3 to 0 and $P_1$ the renumbering
3 to 1. For that, every PE claims a new label and this are globally summed to avoid conflicts.
For every conflicting request, we accept the one with the largest weight. In the example above,
we will pick 3 to 1 since $P_1$'s weight is 0.4 and $P_0$'s is 0.3. The PE that did not get claimed
id will request another id in the next round. If all wanted ids are taken, it will pick one
at random. We repeat until every PEs has picked an id.

\paragraph*{Result:} The idea is implemented in aux::redistributeFromPartition. For random partitions
it works, but it does not do anything for k-means. That is, every PE claims its own id.

After the hilbert curve sorting, every PE has points that are local, i.e., $\forall i \in P_j,
\text{localPart}(i)=j$. After k-means, PEs can have points from different blocks but usually
the majority of their local points still belong to them. For the renumbering to work, at least
2 PEs must claim a different id but this rarely happens (I noticed that most blocks still own
$50-70\%$ of their points after k-means). This makes sense. I did not test that in detail, but in
a few graph I tried, every PE claimed its own id, so no renumbering occurred. Maybe with smaller 
blocks this can have some effect.



\end{document}
